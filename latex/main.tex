
\documentclass{article}
\usepackage{amsmath}
\usepackage{geometry}
\usepackage{graphicx} % Required for inserting images
\usepackage{listings}
\usepackage{color}
\usepackage{verbatim} 
\usepackage{fancyvrb}
\usepackage{fvextra}

\definecolor{dkgreen}{rgb}{0,0.6,0}
\definecolor{gray}{rgb}{0.5,0.5,0.5}
\definecolor{mauve}{rgb}{0.58,0,0.82}

\lstset{frame=tb,
  language=Verilog,
  aboveskip=3mm,
  belowskip=3mm,
  showstringspaces=false,
  columns=flexible,
  basicstyle={\small\ttfamily},
  numbers=none,
  numberstyle=\tiny\color{gray},
  keywordstyle=\color{blue},
  commentstyle=\color{dkgreen},
  stringstyle=\color{mauve},
  breaklines=true,
  breakatwhitespace=true,
  tabsize=3
}

\title{Digital Lab Report}
\author{Xiaokun Du}
\date{February 2023}

\begin{document}

% Creates a title based on the \title, \author, and \date provided
\maketitle


\section{Arithmetic logic unit (ALU)}

Write, as a SystemVerilog module with the name alu, a behavioural description of an arithmetic logic unit. \\
The data inputs, SrcA and SrcB, and the data output, ALUResult, are 8-bit vectors. The ALUControl input is a 2-bit vector. \\
The 1-bit output flag Zero = 1 if ALUResult == 0, else Zero = 0. The ALU carries out bitwise logical operations, and addition and subtraction operations, as specified in the table below.
\subsection*{Module Code}
\begin{lstlisting}
module alu(input logic [7:0] SrcA,
            input logic [7:0] SrcB,
            input logic [1:0] ALUControl,
            output logic [7:0] ALUResult,
            output logic Zero);
always_comb
case (ALUControl)
2'b00 : ALUResult = SrcA & SrcB;
2'b01 : ALUResult = SrcA | SrcB;
2'b10 : ALUResult = SrcA + SrcB;
2'b11 : ALUResult = SrcA - SrcB;
default : ALUResult = 8'bx;
endcase

assign Zero = (ALUResult == 8'b0);
endmodule
\end{lstlisting}

\subsection*{Testbench Code}
\begin{lstlisting}
`timescale 1ns/1ps 
`include "alu.sv"

module alu_tb;
logic [7:0] t_SrcA, t_SrcB;
logic [1:0] t_ALUControl;
logic [7:0] t_ALUResult;
logic t_Zero;

alu uut (t_SrcA, t_SrcB, t_ALUControl, t_ALUResult, t_Zero);

initial begin
    $dumpfile("alu_tb.vcd"); 
    $dumpvars(0, alu_tb);
    // Stimulus generator
    t_SrcA = 8'h05; t_SrcB = 8'h0A;
    t_ALUControl = 2'b00; #20;
    t_ALUControl = 2'b01; #20;
    t_ALUControl = 2'b10; #20;
    t_ALUControl = 2'b11; #20;
end

initial begin // Response monitor
    $monitor ("t_ALUControl = %b t_SrcA = %h t_SrcB = %h t_ALUResult = %b t_Zero = %d",t_ALUControl, t_SrcA, t_SrcB, t_ALUResult, t_Zero);
end
endmodule

\end{lstlisting}

\subsection*{Simulations}
The simulation result using Icarus Verilog is as following:
\begin{Verbatim}
VCD info: dumpfile alu_tb.vcd opened for output.
t_ALUControl = 00 t_SrcA = 05 t_SrcB = 0a t_ALUResult = 00000000 t_Zero = 1
t_ALUControl = 01 t_SrcA = 05 t_SrcB = 0a t_ALUResult = 00001111 t_Zero = 0
t_ALUControl = 10 t_SrcA = 05 t_SrcB = 0a t_ALUResult = 00001111 t_Zero = 0
t_ALUControl = 11 t_SrcA = 05 t_SrcB = 0a t_ALUResult = 11111011 t_Zero = 0
\end{Verbatim}
The simulation results using GTKWave is as following:

\end{document}
